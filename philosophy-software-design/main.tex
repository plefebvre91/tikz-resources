%% beamerthemeImperialPoster v1.0 2016/10/01
%% Beamer poster theme created for Imperial College by LianTze Lim (Overleaf)
%% LICENSE: LPPL 1.3
\documentclass[xcolor={table}]{beamer}

\usepackage[size=a4,orientation=portrait,scale=2.3]{beamerposter}
\usetheme{ImperialPoster}
\usepackage{caption}
\usepackage{subcaption}
\usepackage{caption}
\usepackage{amssymb, amsmath}
\usepackage{tikz}
\usetikzlibrary{decorations.text,calc,arrows.meta}

\usecolortheme{ImperialWhite} 

% Redefines the caption setup of the figures environment in the beamer class.
\captionsetup[figure]{labelformat=empty}

\title{A Philosophy of Software Design}
\author{Authors \mainauthor{John Ousterhout}}

\begin{document}
\begin{frame}[fragile=singleslide,t]
  \centering

  \maketitle

  \begin{columns}[onlytextwidth,T]
    \begin{column}{.47\textwidth}


      \begin{block}{Design Complexity}
        Complexity is anything related to the structure of a software that makes it hard to understand or modify. It can be defined as:
        \begin{equation*}
          C = \sum_{p} c_{p} \cdot t_{p}
        \end{equation*}
        Where $p$ is the number of components, $c_p$ the complexity of each component, et $t_p$ the fraction time accorded to it.
      \end{block}

      
      \begin{block}{Nature of complexity}
        \textbf{Depedencies} and \textbf{obscurity} causes complexity. It can be evaluated by questionning the 3 points:
        \begin{itemize}
        \item \textbf{Change amplification}: amount of code affected by each design decision;
        \item \textbf{Cognitive load}: how much a developer need to know to complete a task;
        \item \textbf{Unknown unknows}: obviousness of which piece of code must be modified to complete a task.
        \end{itemize}
      \end{block}
      

      \begin{block}{Strategic vs. Tactical}
        \begin{sidefigure}
          \centering \scalebox{0.7}{\begin{tikzpicture}
  \coordinate (O) at (0,0);
  \draw[>={Triangle[length=1mm,width=3mm]},->] (O) -- (5,0);
  \draw[>={Triangle[length=1mm,width=3mm]},->] (O) -- (0,4);
  
  \node[text=black, rotate=90] at (-0.5, 1.5) {Progress};
  \draw[text=black] (2, -0.4) node{Time};

  \draw [domain=0:4, dashed] plot (\x,{ln(3*\x+1)});
  \node[text=black, rotate=14] at (3.5, 2) {Tactical};

  \draw [domain=0:4] plot (\x,{\x});
  \node[text=black, rotate=45] at (2.5, 3) {Strategic};
\end{tikzpicture}
}
          \caption {Tactical programming make things work, quickly. Strategic programming invest time on design.}
        \end{sidefigure}
        At the beginning, a tactical approach to programming will make progress more quickly than a strategic approach. However, complexity accumulates more rapidly under the tactical approach, wich reduces productivity.
      \end{block}


    \end{column}
    \begin{column}{.47\textwidth}

      \begin{block}{Deep vs. Shallow}

        \begin{figure}
          \centering
          \begin{subfigure}[t]{0.47\textwidth}
            \centering \scalebox{0.5}{\begin{tikzpicture}
  \coordinate (O) at (0,0);
  \draw (O) rectangle (5,2);
  \draw[line width=2mm] (0,2) -- (5,2);
  \node[text=black] at (2.5, 2.5) {<Interface>};
  \node[text=black] at (2.5, 1) {Functionnalities};
\end{tikzpicture}
}
            \caption{Shallow module}
          \end{subfigure}
          \hfill
          \begin{subfigure}[t]{0.47\textwidth}
            \centering
            \centering \scalebox{0.4}{\begin{tikzpicture}
  \coordinate (O) at (0,0);
  \draw (O) rectangle (3,4);
  \draw[line width=2mm] (0,4) -- (3,4);
  \node[text=black] at (1.5, 4.5) {<Interface>};
  \node[text=black] at (1.5, 2) {Functionnalities};
\end{tikzpicture}
}
            \caption{Deep module}
          \end{subfigure}
        \end{figure}

        \textbf{Deep modules} have a simple interface and powerful functionnalities, \textbf{shallow modules} have complex interface, not much functionnality and hide does not hide complexity.
      \end{block}      
      \begin{block}{Together or Apart ?}
        Modules, classes or functions should be together if:
        \begin{itemize}
        \item information is shared,
        \item it simplifies interface,
        \item it avoids ressources duplication.
        \end{itemize}
        Keep a separation between specialized and general entities.
      \end{block}


      \begin{block}{Comments}
        Comments should capture information that was in the mind of the designer and could not be represented in the code.
      \end{block}

      \begin{block}{Naming}
        ~\\
        \emph{``The greater the distance between a name's declaration and its uses, the longer the name should be.''}
        \begin{flushright}Andrew Gerrand\end{flushright}
      \end{block}
            
    \end{column}
  \end{columns}
\end{frame}
\end{document}
